\noindent


% \textbf{Problem statement}: Introduce the problem that you solved

% % remove vspace in your final version
% \vspace{20 pt}

% \\\textbf{Research methods}: Briefly describe your (novel, innovative) methods. how did you approach to the problem? How did you solve it?

% % remove vspace in your final version
% \vspace{10 pt}

% \\\textbf{Main results}: Present your main (quantified) results. When applicable, present the results in the following format: our work improved ... by ..\% or $x$ times.

%%%%%%%%%%%%%%%%%%%%%%%%%%%%%%%%%%%%%%%%%%%%%

While many researchers work in the field of Digital Twins (DT), the focus mostly lies on parts of the DT. On one side, there are software engineers, who try to model physical devices or assets and system behavior into a digital representation using model-driven engineering. On the other side, there are cloud researchers, who try to create tools to allocate cloud resources and services, optimize costs, energy consumption and usability or automatize and simply deployments. However, these efforts remain disconnected, and a unified approach is lacking.\\

This thesis aims to bridge this gap by combining three projects into an end-to-end solution: The Eclipse AASX-Server [..] to model and standardize the digital representation of a physical asset, a Scheduler project [..] to define the environment of the asset and optimize costs and cloud resource usage and a Deployer project [..] for automatizing the deployment of cloud services. \\

The main challenge addressed is the integration of outputs of the AASX-Server and the Scheduler to generate a functional multi-layer cloud architecture, deployable across multiple cloud providers. The generated cloud architecture for the asset will be in JSON/YAML format and represents an intermediate step between three projects. This work presents a scalable approach for creating Digital Twins within cloud environments, advancing the integration of physical assets and cloud infrastructure.
